\section{Experiment}

Our analysis is composed of results from the three methodologies. Taken together, we expect that these three mechanisms provide a robust indication of which features affect patient outcomes and which are unimportant, and provide useful conclusions to the questions posed in this paper. %We exclude the income feature, as the values of this feature are out of scale with the rest of the parameters. This implies that under our model, the probability of death is assumed to be independent of income. 


We begin our analysis by generating synthetic data according to the procedure described in Section \ref{sec:synthetic_data}. For Task 1, this dataset $\Omega_1$ is then used to validate the methodologies with respect to understanding the impact of comorbidities and vaccination on the chance of death. Having established the efficacy of the three procedures, we then perform methodologies 1, 2, and three on the observational data to determine which comorbidities and vaccines have an effect on the chance of death.

Our experiment for answering the questions of Task 2 is much the same, again generating a synthetic dataset $\Omega_2$

%with calculating the autocorrelation matrix of the synthetic data

%In the part of our analysis using logistic regression models, we first train a logistic regression model on the synthetic data, using cross validation with $10$ folds, and a regularization parameter using the L2 norm. The trained model correctly learns higher coefficient values for the features correlated with death, and coefficients close to zero for the genes which are uncorrelated with the target. From this, we conclude that a logistic regression classifier trained with cross-validation and a regularization penalty can effectively learn from data wherein some variables are predictive of the target (death), and thereby we can identify features which affect the probability of death. We also perform our calculation of correlation matrices and conditional probabilities on the synthetic data as described in the previous paragraph, again to establish that these processes work as expected for known outcomes.

%After establishing the efficacy of the logistic regression model on synthetic data, we then fit logistic regression models to the observational and treatment data. These models are trained as with the synthetic data, maintaining the number of folds for cross-validation to ensure that the model behavior is the same with respect to identifying relevant features. The model is trained on a balanced set of the data, where we take equal numbers of rows with positive and negative response values when applicable\footnote{Details in GitHub repository}. For Task 1a, the model will identify the 5 most important features relevant to the \textit{Death} outcome. In Task 1b and 1c, the model will identify the impact of vaccines on \textit{Death} as well as other symptoms. For Task 2, the model will learn features identifying the impact of treatments on \textit{Death} and other symptoms. Following these models, we also calculate the conditional probability and correlation matrices of features to determine the effect of features on outcomes in the observational and treatment data. We compare these statistics with the estimations learned by the logistic regression models to determine the impact of each feature on each outcome.

%OLD Results in paragraph below
%The model is trained on a balanced set of the data, where we have equal numbers of deaths and non-deaths, using all of the deaths in the data and a random sample of non-death rows. The resulting model had an accuracy of approximately $0.75$, with relatively few of the features being especially predictive of death. The highest coefficient value in the model was associated with the "No-Taste/Smell" feature, with a value of approximately $1.85$. This compares with the highest feature coefficient of approximately $0.75$ in the synthetic data, associated with the genes which jointly double the probability of death. By comparison to the coefficient values of the features strongly predictive of death in our synthetic data, we also find a noticeably negative association between the vaccines and death, with coefficient values of approximately $-0.55$, $-0.32$, and $-0.55$ for vaccines 1, 2, and 3, respectively. Additionally, a number of genes are also predictive of death, with the 15th gene feature being most predictive with a model coefficient value of $0.2$.

We apply methodology 2 twice in Task 2b, first the explanatories are the same, treatments 1 and 2, where the response is the symptoms from before and after. Then we compare if the conditional probabilities increased or decreased, to infer whether the treatment made the symptom more or less likely.

\label{sec:Experimenting}
