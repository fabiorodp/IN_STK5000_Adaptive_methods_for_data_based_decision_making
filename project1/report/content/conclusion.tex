\section{Conclusion}
\label{sec:conclusion}

To conclude, we summarize the answers to the questions posed in Section \ref{sec:introduction} and illustrated in detail in Section \ref{sec:Results}. Our analysis is composed of results from the three methodologies. Taken together, we expect that these three mechanisms provide a robust indication of which features affect patient outcomes and which are unimportant, and provide useful conclusions to the questions posed in this paper. We first successfully establish the efficacy of these methodologies on synthetic data, demonstrating that they are able to illustrate the relations between features that we defined in the data generation. 

When applied to the real data, our methodologies are able to provide the following conclusions. To our questions (1a.), we are able to successfully predict death or survival given genes and comorbidities in the data. The most explanatory features for predicting death in the $CovidPositive$ population were $Gene_{16}$, $Gene_{63}$, and $Gene_{27}$. Our second questions (1b. and 1c.) are answered with the observation that $Vaccine1$ is the most effective in preventing death, but also that all Vaccines show an ability to decrease the probability of death. We also observe likely side-effects from these vaccines, specifically $Headache$ and $Fever$. The probability of these side-effects from all three vaccines were similar. To our final questions (2), we were unable to determine if any side-effects are caused by the treatments or what they may be. However, our methodologies do predict that $Treatment2$ is very likely to be superior to $Treatment1$ in preventing death.
